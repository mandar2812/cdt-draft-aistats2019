\documentclass[twoside]{article}

\usepackage{aistats2019}
\usepackage{amssymb}
% If your paper is accepted, change the options for the package
% aistats2019 as follows:
%
%\usepackage[accepted]{aistats2019}
%
% This option will print headings for the title of your paper and
% headings for the authors names, plus a copyright note at the end of
% the first column of the first page.

% If you set papersize explicitly, activate the following three lines:
%\special{papersize = 8.5in, 11in}
%\setlength{\pdfpageheight}{11in}
%\setlength{\pdfpagewidth}{8.5in}

% If you use natbib package, activate the following three lines:
%\usepackage[round]{natbib}
%\renewcommand{\bibname}{References}
%\renewcommand{\bibsection}{\subsubsection*{\bibname}}

% If you use BibTeX in apalike style, activate the following line:
%\bibliographystyle{apalike}

\begin{document}

% If your paper is accepted and the title of your paper is very long,
% the style will print as headings an error message. Use the following
% command to supply a shorter title of your paper so that it can be
% used as headings.
%
%\runningtitle{I use this title instead because the last one was very long}

% If your paper is accepted and the number of authors is large, the
% style will print as headings an error message. Use the following
% command to supply a shorter version of the authors names so that
% they can be used as headings (for example, use only the surnames)
%
%\runningauthor{Surname 1, Surname 2, Surname 3, ...., Surname n}

\twocolumn[

\aistatstitle{Causal Dynamic Time Lag: Predicting What \& When}

\aistatsauthor{ M. Chandorkar, C. Furthlener, E. Camporeale and  M. Sebag }

\aistatsaddress{CWI Amsterdam, INRIA Paris Saclay} ]

\begin{abstract}
  We formalize the joint regression task of predicting the magnitude of signals as well as the time delay with respect to their driving phenomena. 
\end{abstract}

\section{Introduction}


\section{Causality in Time Series}

First level headings are all caps, flush left, bold, and in point size
12. Use one line space before the first level heading and one-half line space
after the first level heading.

\subsection{Granger Causality}

Second level headings are initial caps, flush left, bold, and in point
size 10. Use one line space before the second level heading and one-half line
space after the second level heading.

\subsubsection{Previous Work}

Third level headings are flush left, initial caps, bold, and in point
size 10. Use one line space before the third level heading and one-half line
space after the third level heading.

\subsection{CITATIONS, FIGURES, REFERENCES}


\subsubsection{Citations in Text}

Citations within the text should include the author's last name and
year, e.g., (Cheesman, 1985). References should follow any style that
you are used to using, as long as their style is consistent throughout
the paper.  Be sure that the sentence reads correctly if the citation
is deleted: e.g., instead of ``As described by (Cheesman, 1985), we
first frobulate the widgets,'' write ``As described by Cheesman
(1985), we first frobulate the widgets.''  %Be sure to avoid
%accidentally disclosing author identities through citations.

\subsubsection{Footnotes}

Indicate footnotes with a number\footnote{Sample of the first
  footnote.} in the text. Use 8 point type for footnotes. Place the
footnotes at the bottom of the column in which their markers appear,
continuing to the next column if required. Precede the footnote
section of a column with a 0.5 point horizontal rule 1~inch (6~picas)
long.\footnote{Sample of the second footnote.}

\subsubsection{Figures}

All artwork must be centered, neat, clean, and legible.  All lines
should be very dark for purposes of reproduction, and art work should
not be hand-drawn.  Figures may appear at the top of a column, at the
top of a page spanning multiple columns, inline within a column, or
with text wrapped around them, but the figure number and caption
always appear immediately below the figure.  Leave 2 line spaces
between the figure and the caption. The figure caption is initial caps
and each figure should be numbered consecutively.

Make sure that the figure caption does not get separated from the
figure. Leave extra white space at the bottom of the page rather than
splitting the figure and figure caption.
\begin{figure}[h]
\vspace{.3in}
\centerline{\fbox{This figure intentionally left non-blank}}
\vspace{.3in}
\caption{Sample Figure Caption}
\end{figure}

\subsubsection{Tables}

All tables must be centered, neat, clean, and legible. Do not use hand-drawn tables.
Table number and title always appear above the table.
See Table~\ref{sample-table}.

Use one line space before the table title, one line space after the table title,
and one line space after the table. The table title must be
initial caps and each table numbered consecutively.

\begin{table}[h]
\caption{Sample Table Title} \label{sample-table}
\begin{center}
\begin{tabular}{ll}
\textbf{PART}  &\textbf{DESCRIPTION} \\
\hline \\
Dendrite         &Input terminal \\
Axon             &Output terminal \\
Soma             &Cell body (contains cell nucleus) \\
\end{tabular}
\end{center}
\end{table}

\section{Problem Formulation}

\emph{Causal Dynamic Time Lag} is essentially a regression problem with two tasks. Given two time series, 
the causes $x(t)$ and the observed effects $y(t)$, the regression model must learn a mapping $f()$ which
maps each input pattern $x(t_1)$ to an output $y(t_2)$, and a mapping $g()$ which maps the time delay between
the input and output patterns $t_2 = t_1 + g(x(t_1))$. This is formally specified in equations below.

\begin{align}
t & \in \mathbb{R}^{+}\\
x(t) & \in \mathcal{X} \\
f: \mathcal{X} & \rightarrow \mathbb{R}\\
g: \mathcal{X} & \rightarrow \mathbb{R}^{+}\\
\Delta(t) & = g[x(t)] \\
y(t + \Delta(t)) & = f[x(t)]
\end{align}

\begin{align}
t & \in \mathbb{R}^{+}\\
x(t) & \in \mathcal{X} \\
f: \mathcal{X} & \rightarrow \mathbb{R}\\
g: \mathcal{X} & \rightarrow \mathbb{R}^{+}\\
\Delta(t) & = g[x(t)] \\
y(t + \Delta(t)) & = f[x(t)]
\end{align}



\section{Proposed Solution}

In practical time-series applications, one works with sub-sampled or discretized versions of the time 
series $x(t)$ and $y(t)$. The time delay function $g(.)$ can now be recast as a function which for every
input pattern $x(t_i)$, returns a time delay $\delta$ corresponding the the time step $t_i + \delta$ when
the effect $y(t_i + \delta)$ is observed.

For practical purposes one must define for every time step $t$, a \emph{causal time window} $[t+\ell, t+\ell+h]$, 
within which the model searches for probable temporal causal links.

Our proposed model then produces the following predictions:

\begin{enumerate}
\item Targets $y(t+\ell), \cdots, y(t+\ell+h-1)$
\item Time Lag Probabilities $\hat{p}(t+\ell), \cdots, \hat{p}(t+\ell+h-1)$
\end{enumerate}

The model thus tries to learn a predictor for each lagged output $y(t+i)$ in the 
causal window $[t+\ell, t+\ell+h]$, and simultaneosly supplies a probability distribution 
over each time step of the causal window, $\hat{p}(t+i)$.

The distribution $\hat{p}(t+i), i \in {\ell, \cdots, \ell+h-1}$ represents the 
likelihood of a causal link between $x(t)$ and $y(t+i)$. Since the model looks
for causal links in the finite window $[t+\ell, t+\ell+h)$, we have 
$\sum^{\ell+h-1}_{i = \ell}{\hat{p}(t + i)} = 1$.


\section{Loss Function}

In order to train time lag based models, one must balance two incentives.

\begin{enumerate}
    \item Generate accurate predictions for time window $y(t+\ell), \cdots, y(t+\ell+h-1)$
    \item Learn the time lag structure.
\end{enumerate}

These incentives have very different contexts. We have access to measurements of the output
time series $y(t)$, but that is generally not the case for the time lag structure. Although
it is possible that certain data sets may have patterns annotated with time lag information, 
this is not the norm.

We express the loss function as a sum of two terms (equation \ref{eq:loss}). The first; which computes a weighted $L_2$ 
error between the predictions $\hat{y}^{(m)}_{i}$ and the targets $y^{(m)}_{i}$. The second term
$\mathcal{J}(y^{(1:M)}, \hat{y}^{(1:M)}, \hat{p}^{(1:M)})$ must penalise the predicted probabilities
$\hat{p}^{(m)}$ in a meaningful manner.

\begin{equation}\label{eq:loss}
\begin{aligned}
&\mathcal{L}(y^{(1:M)}, \hat{y}^{(1:M)}, \hat{p}^{(1:M)}) = \\
&\lambda_1 \sum_{i,m}{\frac{1}{2M} (y^{(m)}_{i} - \hat{y}^{(m)}_{i})^2 (1 + \hat{p}^{(m)}_i)} \\ 
&+ \\ 
&\lambda_2 \mathcal{J}(y^{(1:M)}, \hat{y}^{(1:M)}, \hat{p}^{(1:M)})
\end{aligned}
\end{equation}

\subsection{Target Probability}

The term $\mathcal{J}(y^{(1:M)}, \hat{y}^{(1:M)}, \hat{p}^{(1:M)})$ penalizes the predicted probabilities $\hat{p}^{(1:M)}$, for deviation from some chosen \textit{target probability}.

From the intuitions of Granger causality, we use the concept of causality as predictability, we can thus characterize the \emph{target probability}, $\widetilde{p}$ for a time window $[t+\ell, t+\ell+h)$ in the following manner: 
The lagged output $y(t+i)$ which has greater predictability given $x(t)$, is a more likely causal link. 

\begin{equation}
\widetilde{p}_{i}^{(m)} = \frac{exp \left( \frac{1}{T} (y_{i}^{(m)} - \hat{y}_{i}^{(m)})^{2} \right)}{\sum_{i}{exp \left( \frac{1}{T} (y_{i}^{(m)} - \hat{y}_{i}^{(m)})^{2} \right)}} 
\end{equation}

\begin{align}
& \mathcal{J}(y^{(1:M)}, \hat{y}^{(1:M)}, \hat{p}^{(1:M)}) = \sum_{m = 1}^{M}{\frac{1}{M} \mathcal{H}(\hat{p}^{(m)}, \widetilde{p}^{(m)})} \\
& \mathcal{H}(p, q) = \sqrt{\sum_{i}{(\sqrt{p_i} -  \sqrt{q_i})^2}}\\
\end{align}



\section{Experiments}

For the camera-ready paper, if you are using \LaTeX, please make sure
that you follow these instructions.  (If you are not using \LaTeX,
please make sure to achieve the same effect using your chosen
typesetting package.)

\begin{enumerate}
    \item Download \texttt{fancyhdr.sty} -- the
    \texttt{aistats2019.sty} file will make use of it.
    \item Begin your document with
    \begin{flushleft}
    \texttt{\textbackslash documentclass[twoside]\{article\}}\\
    \texttt{\textbackslash usepackage[accepted]\{aistats2019\}}
    \end{flushleft}
    The \texttt{twoside} option for the class article allows the
    package \texttt{fancyhdr.sty} to include headings for even and odd
    numbered pages. The option \texttt{accepted} for the package
    \texttt{aistats2019.sty} will write a copyright notice at the end of
    the first column of the first page. This option will also print
    headings for the paper.  For the \emph{even} pages, the title of
    the paper will be used as heading and for \emph{odd} pages the
    author names will be used as heading.  If the title of the paper
    is too long or the number of authors is too large, the style will
    print a warning message as heading. If this happens additional
    commands can be used to place as headings shorter versions of the
    title and the author names. This is explained in the next point.
    \item  If you get warning messages as described above, then
    immediately after $\texttt{\textbackslash
    begin\{document\}}$, write
    \begin{flushleft}
    \texttt{\textbackslash runningtitle\{Provide here an alternative
    shorter version of the title of your paper\}}\\
    \texttt{\textbackslash runningauthor\{Provide here the surnames of
    the authors of your paper, all separated by commas\}}
    \end{flushleft}
    Note that the text that appears as argument in \texttt{\textbackslash
      runningtitle} will be printed as a heading in the \emph{even}
    pages. The text that appears as argument in \texttt{\textbackslash
      runningauthor} will be printed as a heading in the \emph{odd}
    pages.  If even the author surnames do not fit, it is acceptable
    to give a subset of author names followed by ``et al.''

    \item Use the file sample\_paper.tex as an example.

    \item The camera-ready versions of the accepted papers are 8
      pages, plus any additional pages needed for references.

    \item If you need to include additional appendices,
      you can include them in the supplementary
      material file.

    \item Please, don't change the layout given by the above
      instructions and by the style file.

\end{enumerate}

\subsubsection*{Acknowledgements}

Use the unnumbered third level heading for the acknowledgements.  All
acknowledgements go at the end of the paper.

\subsubsection*{References}

References follow the acknowledgements.  Use an unnumbered third level
heading for the references section.  Any choice of citation style is
acceptable as long as you are consistent.  Please use the same font
size for references as for the body of the paper---remember that
references do not count against your page length total.

\begin{thebibliography}{}
\setlength{\itemindent}{-\leftmargin}
\makeatletter\renewcommand{\@biblabel}[1]{}\makeatother
\bibitem{} J.~Alspector, B.~Gupta, and R.~B.~Allen (1989).
    \newblock Performance of a stochastic learning microchip.
    \newblock In D. S. Touretzky (ed.),
    \textit{Advances in Neural Information Processing Systems 1}, 748--760.
    San Mateo, Calif.: Morgan Kaufmann.

\bibitem{} F.~Rosenblatt (1962).
    \newblock \textit{Principles of Neurodynamics.}
    \newblock Washington, D.C.: Spartan Books.

\bibitem{} G.~Tesauro (1989).
    \newblock Neurogammon wins computer Olympiad.
    \newblock \textit{Neural Computation} \textbf{1}(3):321--323.
\end{thebibliography}
\end{document}
